\chapter{总结与未来规划}
\label{cha:7-conclusion}

\section{总结}
\label{sec:conclusion}

  随着深度学习的快速发展与其在图形学、计算机视觉各个领域的广泛应用,也出现了大量的研究着眼于着色这一图形学经典问题,并且取得了与传统方法相比更节省人力且效果也不错的结果。但是作为图像着色的相关问题,视频着色就显得研究热度不高,非学习方法在处理图像着色时有顺带进行了视频着色的,但现在最先进的深度学习图像方法还没有做视频着色的尝试。如果能利用深度学习给黑白视频着色,也是对着色问题的一步重要推进。目前来看这方面还没有市面可见的研究。

  本文训练了多种基于生成对抗模型的图像着色网络,如Encoder-and-Decoder、U-NET等等,使用这些模型在3个不同规模、不同图像数据集上进行了训练,检验了模型的可行性。然后使用效果最好的模型在一个视频数据集上训练了新的网络,用于视频着色。在图像着色以外,为了保证视频着色的空间与时间连续性,本文还提出了平滑算法用于处理帧内与帧间的连续性。此外本文也训练了直接用于视频着色的3D网络,不过经过验证效果不太好。

  在图像着色问题上,本文通过与现有最先进方法的量化比较以及肉眼观察比较,取得了尚可的效果;而在视频着色问题上,通过模型选择与平滑算法的处理,本文的方法在给定数据集上的视频着色取得了一定效果。

\section{未来规划}
\label{sec:future-works}

  本文中的系统最终取得的效果与预期结果还有一定差距,本文期望在以下方面在未来对现在的方法进行改进。

  \begin{enumerate}
    \item 着色的准确度。本文的GAN模型相比现在最好的CNN着色模型还有一定差距,原因归结为GAN模型间接由判别器给出损失使得其在处理涉及多风格的着色问题时比较无力。在未来考虑模型的更换或者对GAN模型的学习能力更适合怎样的问题进行探究。
    \item 图像的分辨率支持。现在本文的模型出于对训练速度与准确度的考虑只支持较低的分辨率,未来需要对模型的输入输出层进行修改,采用上采样下采样等方法使网络能处理任意分辨率的图像。
    \item 更多的评价方式。现在关于图像生成方法比较流行的测试是‘AMT real vs. fake’,鉴于对于一张灰度图而言,其合理的着色显然不是只有真实图像一种可能,均方误差不能容许别的可能。‘AMT real vs. fake’测试的内容是:每次给测试者两张图片,一张是真实彩色图像,另一张用灰度图生成的彩色图像,测试者需要判断哪一张是真的,哪一张是假的,最后通过测试者被生成图片骗过的概率来评价生成方法。现阶段本文的方法由于成本还没有通过这一测试进行评价。
  \end{enumerate}

  论文虽然已经结束,但在视频着色领域的探索还有很长的路要走,希望可以在这个方向做出更大的成绩。