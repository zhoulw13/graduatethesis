\thusetup{
  %******************************
  % 注意:
  %   1. 配置里面不要出现空行
  %   2. 不需要的配置信息可以删除
  %******************************
  %
  %=====
  % 秘级
  %=====
  secretlevel={秘密},
  secretyear={10},
  %
  %=========
  % 中文信息
  %=========
  ctitle={黑白视频自动着色算法},
  cdegree={工学硕士},
  cdepartment={软件学院},
  cmajor={软件工程},
  cauthor={周立旺},
  csupervisor={王斌\ 副教授},
  %cassosupervisor={陈文光教授}, % 副指导老师
  %ccosupervisor={某某某教授}, % 联合指导老师
  % 日期自动使用当前时间,若需指定按如下方式修改:
  % cdate={超新星纪元},
  %
  % 博士后专有部分
  %cfirstdiscipline={计算机科学与技术},
  %cseconddiscipline={系统结构},
  %postdoctordate={2009年7月——2011年7月},
  %id={编号}, % 可以留空: id={},
  %udc={UDC}, % 可以留空
  %catalognumber={分类号}, % 可以留空
  %
  %=========
  % 英文信息
  %=========
  %etitle={An Introduction to \LaTeX{} Thesis Template of Tsinghua University v\version},
  % 这块比较复杂,需要分情况讨论:
  % 1. 学术型硕士
  %    edegree:必须为Master of Arts或Master of Science(注意大小写)
  %             “哲学、文学、历史学、法学、教育学、艺术学门类,公共管理学科
  %              填写Master of Arts,其它填写Master of Science”
  %    emajor:“获得一级学科授权的学科填写一级学科名称,其它填写二级学科名称”
  % 2. 专业型硕士
  %    edegree:“填写专业学位英文名称全称”
  %    emajor:“工程硕士填写工程领域,其它专业学位不填写此项”
  % 3. 学术型博士
  %    edegree:Doctor of Philosophy(注意大小写)
  %    emajor:“获得一级学科授权的学科填写一级学科名称,其它填写二级学科名称”
  % 4. 专业型博士
  %    edegree:“填写专业学位英文名称全称”
  %    emajor:不填写此项
  %edegree={Doctor of Engineering},
  %emajor={Computer Science and Technology},
  %eauthor={Xue Ruini},
  %esupervisor={Professor Zheng Weimin},
  %eassosupervisor={Chen Wenguang},
  % 日期自动生成,若需指定按如下方式修改:
  % edate={December, 2005}
  %
  % 关键词用“英文逗号”分割
  ckeywords={黑白视频, 自动着色, 深度网络,CNN, GAN},
  ekeywords={Black and white video, Automatic colorization, Deep network, CNN, GAN}
}

% 定义中英文摘要和关键字
\begin{cabstract}
  黑白图像着色是个由来已久且研究得较为成熟的课题,这个课题的任务是将输入的黑白图
  像还原出它的色彩信息,是图形学领域的一个经典问题。在深度学习普及之前,普遍使用
  的方法是通过人为指定一些局部颜色信息,然后用最优化的方法补全颜色。而最近几年随
  着深度学习的广泛应用,许多研究也用深度学习的方法在黑白图像着色问题上做到了自动
  化,而不需要人工干涉,得到的结果也非常不错。

  黑白视频着色问题相比图像着色问题更难,相关的研究也更少。传统的通过人工干涉的最
  优化方法有对视频着色的研究,通过将最优化目标由二维扩展到三维上做到;但现在还没
  有考虑时间连续性的全自动的黑白视频着色算法,这就是本文的研究目标。

  在这篇论文中,我训练了多种对黑白图像着色的GAN模型,训练了多种对视频关键帧着色
  的GAN模型,处理了视频关键帧之间黑白帧的着色问题。
\end{cabstract}

% 如果习惯关键字跟在摘要文字后面,可以用直接命令来设置,如下:
% \ckeywords{\TeX, \LaTeX, CJK, 模板, 论文}

\begin{eabstract}
   As one of the basic tasks in computer graphics, Black and white image colorization has been a long exist and fully researched task, which aims to restore color information of a gray input image. Before deep learning has been widely applied, a traditional solution to this problem is using optimization, with local color information manually offered, called user-guided colorization. while with the development of deep learning these years, many new methods have been raised, which deals with automatic image colorization and reaches the state-of-art.

   Comparing to image colorization, Black and white video colorization is a harder and less researched problem. Traditional way of optimization with manual hints can do this by applying 2D objective function to 3D. While there are no mature fully automatic methods of video colorization concerning about time consistency, this paper tries to find a solution.

   In this paper, I train many image colorization and video key-frames colorization models based on GAN. Except for that, I also deal with task of colorizing inter-frames between key-frames.

\end{eabstract}

% \ekeywords{\TeX, \LaTeX, CJK, template, thesis}
