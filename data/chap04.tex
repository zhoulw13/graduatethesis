\chapter{黑白视频着色}
\label{cha:4-video-color}

  在2D模型上验证了可行性后,就要将模型扩展到3D上处理黑白视频着色问题。我们使用GAN做视频着色的想法来源于3D-GAN的论文~\cite{DBLP:conf/nips/0001ZXFT16}。这篇论文的目的是从一个随机变量生成3D像素空间的模型,有趣的是他们还训练了一个生成器为VAE模型的网络,用于将2D图片转换为3D模型。对于我们的问题而言,从2D扩展到3D不是在空间上,而是在时间上。另外我们并不是将整个视频作为3D数据输入,因为每个视频的长短不一样,网络无法处理帧数不一样的各个视频,所以我们采取等间距采样获取一定数量的关键帧,将这部分关键帧通过网络着色后再用非学习的方法给剩余未着色的帧着色。

  我们的3D着色模型判别器与生成器均分别从2D对应的模型扩展而来,下面介绍我们的网络结构以及训练参数。

\section{判别器网络}
\label{sec:4-d-net}
  
  判别器网络直接从图片着色网络扩展而来:使用3D的卷积函数和3D的BN函数;之前的卷积核是2D的,大小为4x4,扩展到3D后,大小为4x4x4,同样步幅由2x2变为2x2x2,边缘填充由1x1变为1x1x1。卷积层及其对应的BN层和LeakyReLU层还是4层。最后经过一个卷积核为1x4x4\footnote{卷积核大小为1x4x4的原因在第\ref{sec:4-train}节解释}的卷积层得到判别器给出的图像为真实的概率。
  
\section{生成器网络}
\label{sec:4-g-net}

  我们的生成器网络选择从U-NET扩展到3D,因为在黑白图像着色阶段被证明U-NET是最好的结构。这个网络的结构如下:输入64x64x8的单通道图像序列,经过6层卷积核4、步幅2、边缘填充1的卷积层及其对应的BN层和LeakyReLU层,得到1x1x1x256的特征向量,然后经过6层卷积核卷积核4、步幅2、边缘填充1的反卷积层,得到64x64x8的双通道图像,再经过一层tanh激励函数后得到输出色度图像序列。其中在解码的反卷积阶段,前面对应的编码卷积层的输出特征都添加到反卷积输入中,以帮助空间信息的恢复。

\section{训练参数}
\label{sec:4-train}

  黑白视频着色训练的相关参数如表~\ref{tab:4-video-train}。

  \begin{table}[h]
    \centering
    \begin{minipage}[t]{0.8\linewidth}
    \caption{黑白视频着色GAN训练参数}
    \label{tab:4-video-train}
      \begin{tabularx}{\linewidth}{lXX}
        \toprule[1.5pt]
        {\heiti 参数名} & {\heiti 描述} & {\heiti 参数值} \\\midrule[1pt]
        image size & 图像分辨率 & 64x64 \\
        batch size & 批处理大小 & 64 \\
        frame size & 单视频截取帧数 & 8 \\
        epochs & 重复训练次数 & 200 \\
        ndf & 判别器第一层特征通道数 & 64 \\
        ngf & 生成器第一层特征通道数 & 64 \\
        learn rate D & 判别器学习率 & $5\times10^{-5}$ \\
        learn rate G & 生成器学习率 & $5\times10^{-5}$ \\
        D iterations & 每迭代一次生成器,判别器的迭代次数 & 5 \\
        clamp & 固定判别器参数的常数 & $\pm0.01$ \\
        optimizer & 优化函数 & RMSprop \\
        \bottomrule[1.5pt]
      \end{tabularx}
    \end{minipage}
  \end{table}

  视频着色与图像着色的差别就是多了一个单视频截取帧数的参数,在这里我们取值为8,即对于训练数据集中的每个视频,我们等间距的取出8帧图像代表这个视频,输入给着色网络。帧数上的分辨率小于图像空间上的分辨率就会导致一个问题,即使用卷积核为4的卷积时,帧数维度会先于空间维度压缩到大小为1,继续卷积会出现问题,所以在帧数维度压缩到1后,我们在帧数维度上的卷积层,设置为卷积核为3,步幅为1,边缘填充为1,这样可以保持帧数维度上1的分辨率不变。这也是我们在判别器最后一层卷积层使用了1x4x4的卷积核的原因。